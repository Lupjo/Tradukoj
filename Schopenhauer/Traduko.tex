\documentclass{book}
\usepackage[utf8x]{inputenc}
\usepackage[greek,english]{babel}
\usepackage{xcolor}
\newcommand{\de}[1]{{\color{red}#1}\\}
%\newcommand{\de}[1]{ }
\newcommand{\eo}[1]{#1\\}
%\newcommand{\eo}[1]{ }
\newcommand{\eble}[1]{{\color{blue}#1}}
%\newcommand{\eble}[1]{#1}

\title{La arto ŝajne pravi}
\author{Arthur Schopenhauer\\Tradukis Louis Noizet}
\date{}

\begin{document}
\maketitle

\chapter*{Eristiska Dialektiko}

\de{Eristische Dialektik ist die Kunst zu disputieren, und zwar so zu
disputieren, daß man Recht behält, also per fas et nefas [mit Recht wie
mit Unrecht]. }
\eo{Eristika dialektiko estas la arto disputi tiamaniere, ke oni ŝajne
pravas, do \textit{per fas et nefas} [kun pravo kaj malpravo].}
\de{Man kann nämlich in der Sache selbst objective Recht haben und doch
in den Augen der Beisteher, ja bisweilen in seinen eignen, Unrecht
behalten.}
\eo{Oni nome povas fakte mem \textit{objektivan} pravon havi, kaj dume,
laŭ la okuloj de la spektatoroj, eĉ de tempo al tempo laŭ la siaj, ŝajne
malpravi.}
\de{Wann nämlich der Gegner meinen Beweis widerlegt, und dies als
Widerlegung der Behauptung selbst gilt, für die es jedoch andre Beweise
geben kann; in welchem Fall natürlich für den Gegner das Verhältnis
umgekehrt ist: er behält Recht, bei objektivem Unrecht. }
\eo{Kiam la kontraŭulo nome mia pruvo refutas, kaj valoras tion kiel
refuto de la aserto mem, por tio tamen povas ekzisti alia pruvo; tiam
kompreneble estas la rilato inversigita por la kontraŭulo: li ŝajne
pravas per objektiva malpravo.}
\de{Also die objektive Wahrheit eines Satzes und die Gültigkeit desselben
in der Approbation der Streiter und Hörer sind zweierlei. }
\eo{Do la objektiva vereco de propozicio kaj la valideco de tio laŭ la
aprobo de la batalisto kaj aŭskultantoj estas apartaj.}
\de{(Auf letztere ist die Dialektik gerichtet.) }
\eo{(Al tio celas la dialektiko.)}

\de{Woher kommt das? – Von der natürlichen Schlechtigkeit des
menschlichen Geschlechts. }
\eo{De kie venas tio? – De la \eble{naturala aĉeco de la homa naturo}.}
\de{Wäre diese nicht, wären wir von Grund aus ehrlich, so würden wir bei
jeder Debatte bloß darauf ausgehn, die Wahrheit zu Tage zufördern, ganz
unbekümmert ob solche unsrer zuerst aufgestellten Meinung oder der des
Andern gemäß ausfiele: dies würde gleichgültig, oder wenigstens ganz und
gar Nebensache sein. }
\eo{Sen tio, ni estus tute honesta, do en iu debato, ni nur celus malkaŝi
la veron, ne zorgante ĉu ĉi tio validas nian unue konceptitan opinion aŭ
la opinion de la alio: ĝi estus al ni indiferenta, aŭ almenaŭ akcesora.}
\de{Aber jetzt ist es Hauptsache. }
\eo{Sed fakte, tio estas nepra.}
\de{Die angeborne Eitelkeit, die besonders hinsichtlich der
Verstandeskräfte reizbar ist, will nicht haben, daß was wir zuerst
aufgestellt, sich als falsch und das des Gegners als Recht ergebe. }
\eo{La \eble{denaska} vantemo, kiun aparte incitas atako al nia
intelekta potenco, ne akceptas, ke nia unua opinio riveliĝas kiel
malvera, kaj tio de la kontraŭulo kiel vera.}
\de{Hienach hätte nun zwar bloß jeder sich zu bemühen, nicht anders als
richtig zu urteilen: wozu er erst denken und nachher sprechen müßte. }
\eo{Do ĉiu ja nur devus peni, ne juĝi alie ol vere: por tio oni devus
unue pripensi kaj poste paroli.}
\de{Aber zur angebornen Eitelkeit gesellt sich bei den Meisten
Geschwätzigkeit und angeborne Unredlichkeit. }
\eo{Sed al la \eble{denaska} vantemo kuniĝas plejofte babilaĉo kaj
\eble{denaska} malhonesteco.}
\de{Sie reden, ehe sie gedacht haben, und wenn sie auch hinterher merken,
daß ihre Behauptung falsch ist und sie Unrecht haben; so soll es doch
scheinen, als wäre es umgekehrt. }
\eo{Ili parolas antaŭ ili pensas, kaj eĉ kiam ili ankaŭ poste rimarkas,
ke ilia aserto malveras kaj, ke ili malpravas; tiam ili tamen ŝajnigas ke
estas la kontraŭo.}
\de{Das Interesse für die Wahrheit, welches wohl meistens bei Aufstellung
des vermeintlich wahren Satzes das einzige Motiv gewesen, weicht jetzt
ganz dem Interesse der Eitelkeit: wahr soll falsch und falsch soll wahr
scheinen.}
\eo{La intereso pri vereco, kio estis la sola motivo por la vortigo de la
supozata vera propozicio, cedas nun tute al la intereso de la vantemo:
vera devas falsa, kaj falsa devas vera ŝajni.}

\de{Jedoch hat selbst diese Unredlichkeit, das Beharren bei einem Satz,
der uns selbst schon falsch scheint, noch eine Entschuldigung: oft sind
wir anfangs von der Wahrheit unsrer Behauptung fest überzeugt, aber das
Argument des Gegners scheint jetzt sie umzustoßen; geben wir jetzt ihre
Sache gleich auf, so finden wir oft hinterher, daß wir doch Recht haben:
unser Beweis war falsch; aber es konnte für die Behauptung einen
richtigen geben: das rettende Argument war uns nicht gleich
beigefallen.}
\eo{Tamen, tio malhonesteco, la insisto pri propozicio kio al ni mem jam
malvera ŝajnas, havas ankoraŭ unu ekskuzon: ofte, ni estas al komenco
solide konvinkita pri la veraco de nia aserto, sed la argumento de la
kontraŭulo jam ŝajnas \eble{faligi} ĝin; Se ni nun tuj rezignas, ni ofte
malkovras poste, ke ni ja pravas: nia pruvo estis malĝusta; sed povis
esti korekta pruvo de la aserto: La savanta argumento ne tuj venis al
ni.}
\de{Daher entsteht nun in uns die Maxime, selbst wann das Gegenargument
richtig und schlagend scheint, doch noch dagegen anzukämpfen, im Glauben,
daß dessen Richtigkeit selbst nur scheinbar sei, und uns während des
Disputierens noch ein Argument, jenes umzustoßen, oder eines, unsre
Wahrheit anderweitig zu bestätigen, einfallen werde: hiedurch werden wir
zur Unredlichkeit im Disputieren beinahe genötigt, wenigstens leicht
verführt. }
\eo{De tie naskiĝas en ni la maksimo, eĉ se la kontraŭargumento ŝajnas
korekta kaj frapanta, ankoraŭ kontraŭbatali ĝin, pensanta, ke la
korekteco de tio nur ŝajna estas kaj, ke dum la disputo, ni trovos
argumenton por faligi tion aŭ konfirmi nian verecon: Tial, ni estas
devigita esti malhonesta, almenaŭ tentita.}
\de{Diesergestalt unterstützen sich wechselseitig die Schwäche unsers
Verstandes und die Verkehrtheit unsers Willens. }
\eo{Tiamaniere, la malforteco de nia intelekto kaj la falseco de nia volo
sin reciproke \eble{subtenas}}
\de{Daraus kommt es, daß wer disputiert, in der Regel nicht für die
Wahrheit, sondern für seinen Satz kämpft, wie pro ara et focis [für Heim
\& Herd], und per fas et nefas verfährt, ja wie gezeigt nicht anders
kann.}
\eo{De tio venas, ke kiu disputas, kutime ne batalas por la vereco sed
por sia propozicio, kiel \textit{pro ara et focis} [por hejmo kaj
kovejo], kaj \textit{per fas et nefas}, ja kiel montrita, ne povas esti
alie.}



\de{Jeder also wird in der Regel wollen seine Behauptung durchsetzen,
selbst wann sie ihm für den Augenblick falsch oder zweifelhaft scheint.}
\eo{Do ĉiu volos generale, sian aserton altrudi, eĉ kiam ĝi ŝajnas falsa
aŭ dubinda al liaj okuloj.}
\de{Die Hilfsmittel hiezu gibt einem jeden seine eigne Schlauheit und
Schlechtigkeit einigermaßen an die Hand: dies lehrt die tägliche
Erfahrung beim Disputieren;}
\eo{Oni uzas kiel helpilo por tio sian propran \eble{insidemon} kaj
aĉecon: La ĉiutaga sperto lernigas tion per disputo;}
\de{es hat also jeder seine natürliche Dialektik, so wie er seine
natürliche Logik hat. }
\eo{Do ĉiu havas sian naturalan dialektikon, kiel ĉiu havas sian
naturalan logikon.}
\de{Allein jene leitet ihn lange nicht so sicher als diese. }
\eo{Tamen, jeno ne tiom certe direktas kiom tio.}
\de{Gegen logische Gesetze denken, oder schließen, wird so leicht keiner:
falsche Urteile sind häufig, falsche Schlüsse höchst selten. }
\eo{Pensi aŭ dedukti kontraŭ la logikaj leĝoj ne tia facile
estas: falsaj propozicioj estas oftaj, falsaj deduktoj ege
maloftas.}
\de{Also Mangel an natürlicher Logik zeigt ein Mensch nicht leicht;
hingegen wohl Mangel an natürlicher Dialektik: sie ist eine ungleich
ausgeteilte Naturgabe (hierin der Urteilskraft gleich, die sehr ungleich
ausgeteilt ist, die Vernunft eigentlich gleich). }
\eo{Do al homoj ne facile mankas \eble{naturala} logiko; tamen facile mankas
\eble{naturala} dialektiko: ĝi estas malsame disdonita naturdonaco
(kiel la juĝpovo kio tre \eble{malsame} disdonita estas kaj la racio
\eble{same}).}
\de{Denn durch bloß scheinbare Argumentation sich konfundieren, sich
refutieren lassen, wo man eigentlich Recht hat, oder das umgekehrte,
geschieht oft; und wer als Sieger aus einem Streit geht, verdankt es sehr
oft, nicht sowohl der Richtigkeit seiner Urteilskraft bei Aufstellung
seines Satzes, als vielmehr der Schlauheit und Gewandtheit, mit der er
ihn verteidigte. }
\eo{Do oftas sin \eble{lasi} konfuzi, refuti, per nur ŝajnaj
argumentoj kiam oni vere pravas, aŭ inverse; kaj kio venkas en iu
disputo, tre ofte ŝuldas tion ne al la korekteco de sia \eble{juĝpovo}
sed al la insidemo kaj lerteco kun kiuj li defendas sian aserton.}
\de{Angeboren ist hier wie in allen Fällen das beste: jedoch kann Übung
und auch Nachdenken über die Wendungen, durch die man den Gegner wirft,
oder die er meistens gebraucht, um zu werfen, viel beitragen, in dieser
Kunst Meister zu werden. }
\eo{Tie, estas plej bona kiam la donacoj estas denaskaj: tamen ekzerco
kaj meditado pri la parolturno, kun kiu oni sagas la kontraŭulon, aŭ kiu
li plejofte uzas por sagi, povas multe kontribui al la arto iĝi
majstron.}
\de{Also wenn auch die Logik wohl keinen eigentlich praktischen Nutzen
haben kann: so kann ihn die Dialektik allerdings haben. }
\eo{Do kiam eĉ la logiko povas havi nenia praktikan utilon: tiam la
dialektiko povas dume ĝin havi.}
\de{Mir scheint auch Aristoteles seine eigentliche Logik (Analytik)
hauptsächlich als Grundlage und Vorbereitung zur Dialektik aufgestellt zu
haben und diese ihm die Hauptsache gewesen zu sein. }
\eo{Ŝajnas al mi, ke Aristotelo konceptis sian propran logikon
(analitiko) kiel bazo kaj preparo de la dialektiko kaj, ke estis por li
la \eble{ĉefajo}}
\de{Die Logik beschäftigt sich mit der bloßen Form der Sätze, die
Dialektik mit ihrem Gehalt oder Materie, dem Inhalt: daher eben mußte die
Betrachtung der Form als des allgemeinen der des Inhalts als des
besonderen vorhergehn.}
\eo{La logiko zorgas pri la nura formo de la propozicio, la dialektiko
pri ĝia enhavo aŭ materio: pri ĝia kvintesenco: precize tial, la
konsidero de la formo, kiel \eble{generalaĵo} devis antaŭi tion de la
enhavo, kiel detala.}


\de{Aristoteles bestimmt den Zweck der Dialektik nicht so scharf wie ich
getan: er gibt zwar als Hauptzweck das Disputieren an, aber zugleich auch
das Auffinden der Wahrheit (Topik, I , 2); später sagt er wieder: man
behandle die Sätze philosophisch nach der Wahrheit, dialektisch nach dem
Schein oder Beifall, Meinung Andrer (\textgreek{δοξα}) Topik, I , 1.}
\eo{Aristotelo ne deskribas la celon de la dialektiko tiel precize
\eble{ke/ol} mi faris: li ja indikas la disputon kiel ĉefa celo, sed
samtempe ankaŭ la retrovo de la vero (vd. \textit{Topikoj}, I, 2); Pli
malfrue, li ankoraŭ diras: filozofie, oni traktas la propozicion laŭ la
vero, dialektike, laŭ la ŝajno aŭ aprobo de la opinio de aliaj
(\textgreek{δοξα}, vd. \textit{Topikoj}, I, 2).}
\de{Er ist sich der Unterscheidung und Trennung der objektiven Wahrheit
eines Satzes von dem Geltendmachen desselben oder dem Erlangen der
Approbation zwar bewußt; allein er hält sie nicht scharf genug
auseinander, um der Dialektik bloß letzteres anzuweisen.}
\eo{Li ja konscias pri la malsameco kaj separo inter la objektiva vereco
de propozicio kaj la dirmaniero de tio aŭ la ricevita aprobo; sed li ni
sufiĉe precize distingas ilin, por povi difini dialektikon nur \eble{por
la lasta}.}
\de{Seinen Regeln zu letzterem Zweck sind daher oft welche zum ersteren
eingemengt.}
\eo{Lia reguloj por la lasta tial ofte iom miksas kun la unua.}
\de{Daher es mir scheint, daß er seine Aufgabe nicht rein gelöst hat.}
\eo{Tial ŝajnas al mi, ke li ne klare plenumis lian taskon.}


\de{Aristoteles hat in den Topicis die Aufstellung der Dialektik mit
seinem eignen wissenschaftlichen Geist äußerst methodisch und
systematisch angegriffen, und dies verdient Bewunderung, wenn gleich der
Zweck, der hier offenbar praktisch ist, nicht sonderlich erreicht
worden.}
\eo{Aristotelo atakis en la \textit{Topikoj} la formadon de la dialektiko
kun sia propra scienca spirito, ekstreme metode kaj sisteme, kaj tio
meritas admiron, sed samtempe la celon, kio tie evidente
praktika estas, li ne vere atingis.}
\de{Nachdem er in den Analyticis die Begriffe, Urteile und Schlüsse der
reinen Form nach betrachtet hatte, geht er nun zum Inhalt über, wobei er
es eigentlich nur mit den Begriffen zu tun hat: denn in diesen liegt ja
der Gehalt.}
\eo{Post ke li konsideris la nociojn, verdiktojn kaj konkludojn laŭ iliaj
puraj formoj en la Analitikoj, li transiras nun al la enhavo, kio fakte
rilatas nur al la nocioj: ĉar en ili \eble{ripozas} la enhavon.}
\de{Sätze und Schlüsse sind rein für sich bloße Form: die Begriffe sind
ihr Gehalt.}
\eo{Propozicioj kaj konkludoj estas por ili nura formo: la nocioj estas
ilia enhavo.}
\de{– Sein Gang ist folgender.}
\eo{Lia progresado estas la sekva.}
\de{Jede Disputation hat eine Thesis oder Problem (diese differieren bloß
in der Form) und dann Sätze, die es zu lösen dienen sollen.}
\eo{Iu disputo havas tezon aŭ problemon (tiuj nur malsamas \eble{laŭ} la
formo) kaj poste, propozicioj kiuj devas servi al solvi tion.}
\de{Es handelt sich dabei immer um das Verhältnis von Begriffen zu
einander.}
\eo{Necesas samtempe elmontri rilatoj inter la nocioj.}
\de{Dieser Verhältnisse sind zunächst vier.}
\eo{Tiuj rilatoj estas kvar.}
\de{Man sucht nämlich von einem Begriff, entweder 1. seine Definition,
oder 2. sein Genus, oder 3. sein Eigentümliches, wesentliches Merkmal,
proprium, idion, oder 4. sein accidens, d.i. irgend eine Eigenschaft,
gleichviel ob Eigentümliches und Ausschließliches oder nicht, kurz ein
Prädikat.}
\eo{Oni nome serĉas en iu nocio \eble{unu el la sekvaj} 1. ĝian difinon,
aŭ 2. ĝian genron, aŭ 3. ĝian unikan esencan distingilon
\textit{proprium, idion}, aŭ 4. ĝian \textit{accidens}, t.e. ia eco,
egale ĉu estas unika kaj eksklusiva aŭ ne; resume: predikato.}
\de{Auf eins dieser Verhältnisse ist das Problem jeder Disputation
zurückzuführen.}
\eo{La problemo estas rekonduki ĉiun disputon al unu el tiuj rilatoj.}
\de{Dies ist die Basis der ganzen Dialektik.}
\eo{Tio estas la bazo de la tuta dialektiko.}
\de{In den acht Büchern derselben stellt er nun alle Verhältnisse, die
Begriffe in jenen vier Rücksichten wechselseitig zu einander haben
können, auf und gibt die Regeln für jedes mögliche Verhältnis;}
\eo{En siaj ok libroj, li prezentas nun ĉiuj rilatoj, ke la nocioj
povas havi \eble{kun unu la aliaj} rilate al tiuj kvar punktoj, kaj donas
la regulojn por ĉiu ebla rilato;}
\de{wie nämlich ein Begriff sich zum andern verhalten
müsse, um dessen proprium, dessen accidens,
dessen genus, dessen definitum oder Definition
zu sein: welche Fehler bei der Aufstellung leicht
gemacht werden, und jedesmal was man demnach
zu beobachten habe, wenn man selbst ein solches
Verhältnis aufstellt (\textgreek{κατασκεναζειν}), und was
man, nachdem der andre es aufgestellt, tun könne,
es umzustoßen (\textgreek{ανασκευαζειν}).}
\eo{kiel nocio devus rilati kun alia por esti ĝia proprium, ĝia
accidens, ĝia genus, ĝia definitum aŭ difino: kion eraron facilas la
dirmaniero, kaj ĉiufoje kion oni devus observi, kiam oni prezentas
\eble{tia rilaton} (\textgreek{κατασκεναζειν}), kaj, kiam la alio
prezentis ĝin, kion ni povus fari por \eble{faligi} tion
(\textgreek{ανασκευαζειν}).}
\de{Die Aufstellung
jeder solchen Regel oder jedes solchen allgemeinen
Verhältnisses jener Klassen-Begriffe zu einander
nennt er \textgreek{τοπος} (topos), {\it locus}, und gibt 382
solcher \textgreek{τοποι}: daher Topica.}
\eo{La prezenton de ĉiuj tiuj reguloj aŭ ĉiuj tiuj generalaj rilatoj nomas
li \textgreek{τοπος} (topos), {\it locus}, kaj donas 382 tia
\textgreek{τοποι}: de tie venas la nomo Topica.}
\de{Diesem fügt er noch
einige allgemeine Regeln bei, über das Disputieren
überhaupt, die jedoch lange nicht erschöpfend
sind.}
\eo{Al tio aldonis li ankoraŭ kelkajn generalajn regulojn, ĉiuj koncerne
la disputo, kioj tamen ne plu kompleta estas.}

\de{Der topos ist also kein rein materieller, bezieht
sich nicht auf einen bestimmten Gegenstand, oder
Begriff; sondern er betrifft immer ein Verhältnis
ganzer Klassen von Begriffen, welches unzähligen
Begriffen gemein sein kann, sobald sie zu einander
in einer der erwähnten vier Rücksichten betrach-
tet werden, welches bei jeder Disputation statt
hat.}
\eo{Do la topos ne estas iu materia, ne rilatas al \eble{iu} preciza aĵo,
aŭ nocio; sed ĝi ĉiam koncernas rilaton inter ĉia nocioj, kies
nenombrebla nocioj povas esti kuna, ekde ili estas kunitaj \eble{unu kun
la alia} kun unu el la kvar cititaj \eble{konsideradoj}, kaj tio okazas
en ĉiu disputo.} 
\de{Und diese vier Rücksichten haben wieder
untergeordnete Klassen.}
\eo{Kaj tiuj kvar \eble{konsideradoj} havas ankoraŭ subalternajn
kategoriojn.}
\de{Die Betrachtung ist hier
  also noch immer gewissermaßen formal,}
\eo{Do tiu konsiderado estas tie ankoraŭ, por tiel diri, formala,}
\de{jedoch
  nicht so rein formal wie in der Logik, da sie sich
mit dem Inhalt der Begriffe beschäftigt, aber auf
eine formelle Weise, nämlich sie gibt an, wie der
Inhalt des Begriffs A sich verhalten müsse zu dem
des Begriffs B, damit dieser aufgestellt werden
könne als dessen genus oder dessen proprium
(Merkmal) oder dessen accidens oder dessen
Definition oder nach den diesen untergeordneten
Rubriken, von Gegenteil \textgreek{αντικειμεον}, Ursache
und Wirkung, Eigenschaft und Mangel usw.: und
um ein solches Verhältnis soll sich jede Dispu-
tation drehen.}
\de{Die meisten Regeln, die er nun
eben als topoi über diese Verhältnisse angibt, sind
solche, die in der Natur der Begriffsverhältnisse
liegen, deren jeder sich von selbst bewußt ist, und
auf deren Befolgung vom Gegner er schon von
selbst dringt, eben wie in der Logik, und die es
7leichter ist im speziellen Fall zu beobachten oder
ihre Vernachlässigung zu bemerken, als sich des
abstrakten topos darüber zu erinnern: daher eben
der praktische Nutzen dieser Dialektik nicht groß
ist.}
\de{Er sagt fast lauter Dinge, die sich von selbst
verstehn und auf deren Beachtung die gesunde
Vernunft von selbst gerät.}
\de{Beispiele:}


\de{«Wenn von einem Dinge das genus behauptet
wird, so muß ihm auch irgend eine species dieses
genus zukommen; ist dies nicht, so ist die
Behauptung falsch: z.B. es wird behauptet, die
Seele habe Bewegung; so muß ihr irgend eine
bestimmte Art der Bewegung eigen sein, Flug,
Gang, Wachstum, Abnahme usw. – ist dies nicht,
so hat sie auch keine Bewegung.}
\de{– Also wem
keine Spezies zukommt, dem auch nicht das
genus: das ist der topos.» Dieser topos gilt zum
Aufstellen und zum Umwerfen.}
\de{Es ist der neunte
topos.}
\de{Und umgekehrt: wenn das Genus nicht
zukommt, kommt auch keine Spezies zu: z.B.
Einer soll (wird behauptet) von einem Andern
schlecht geredet haben: – Beweisen wir, daß er
gar nicht geredet hat, so ist auch jenes nicht:
denn wo das genus nicht ist, kann die Spezies
nicht sein.}
\de{Unter der Rubrik des Eigentümlichen,
proprium, lautet der 215. locus so: «Erstlich zum
Umstoßen: wenn der Gegner als Eigentümliches
etwas angibt, das nur sinnlich wahrzunehmen ist,
so ists schlecht angegeben: denn alles Sinnliche
wird ungewiß, sobald es aus dem Bereich der
Sinne hinaus kommt: z.B. er setzt als Eigentüm-
liches der Sonne, sie sei das hellste Gestirn, das
über die Erde zieht: – das taugt nicht: denn wenn
die Sonne untergegangen, wissen wir nicht ob sie
über die Erde zieht, weil sie dann außer dem
Bereich der Sinne ist.}
\de{– Zweitens zum Aufstellen:
das Eigentümliche wird richtig angegeben, wenn
ein solches aufgestellt wird, das nicht sinnlich
erkannt wird, oder wenn sinnlich erkannt, doch
notwendig vorhanden: z.B. als Eigentümliches
der Oberfläche werde angegeben, daß sie zuerst
gefärbt wird; so ist dies zwar ein sinnliches
8Merkmal, aber ein solches, das offenbar allezeit
vorhanden, also richtig.» – Soviel um Ihnen einen
Begriff von der Dialektik des Aristoteles zu geben.}
\de{Sie scheint mir den Zweck nicht zu erreichen: ich
habe es also anders versucht.}
\de{Cicero’s Topica sind
eine Nachahmung der Aristotelischen aus dem
Gedächtnis: höchst seicht und elend; Cicero hat
durchaus keinen deutlichen Begriff von dem, was
ein topus ist und bezweckt, und so radotiert er
ex ingenio [aus freier Erfindung] allerhand Zeug
durcheinander, und staffiert es reichlich mit
juristischen Beispielen aus.}
\de{Eine seiner schlech-
testen Schriften.}


\de{Um die Dialektik rein aufzustellen muß man,
unbekümmert um die objektive Wahrheit (welche
Sache der Logik ist), sie bloß betrachten als die
Kunst, Recht zu behalten, welches freilich um so
leichter sein wird, wenn man in der Sache selbst
Recht hat.}
\de{Aber die Dialektik als solche muß bloß
lehren, wie man sich gegen Angriffe aller Art,
besonders gegen unredliche verteidigt, und eben
so wie man selbst angreifen kann, was der Andre
behauptet, ohne sich selbst zu widersprechen und
überhaupt ohne widerlegt zu werden.}
\de{Man muß
die Auffindung der objektiven Wahrheit rein tren-
nen von der Kunst, seine Sätze als wahr geltend
zu machen: jenes ist [Aufgabe] einer ganz andern
\textgreek{πραγματεια} [Betätigung], es ist das Werk der
Urteilskraft, des Nachdenkens, der Erfahrung,
und gibt es dazu keine eigne Kunst; das zweite
aber ist der Zweck der Dialektik.}
\de{Man hat sie
definiert als die Logik des Scheins: falsch: dann
wäre sie bloß brauchbar zur Verteidigung falscher
Sätze; allein auch wenn man Recht hat, braucht
man Dialektik, es zu verfechten, und muß die
unredlichen Kunstgriffe kennen, um ihnen zu
begegnen; ja oft selbst welche brauchen, um den
Gegner mit gleichen Waffen zu schlagen.}
\de{Dieser-
halb also muß bei der Dialektik die objektive
Wahrheit bei Seite gesetzt oder als akzidentell
betrachtet werden: und bloß darauf gesehn
werden, wie man seine Behauptungen verteidigt
9und die des Andern umstößt; bei den Regeln hiezu
darf man die objektive Wahrheit nicht berück-
sichtigen, weil meistens unbekannt ist, wo sie
liegt: 8 oft weiß man selbst nicht, ob man Recht
hat oder nicht, oft glaubt man es und irrt sich, oft
glauben es beide Teile: denn veritas est in puteo
[Die Wahrheit ist in der Tiefe] (\textgreek{εν βυθῳ ἡ
αληθεια}, Demokrit); beim Entstehn des Streits
glaubt in der Regel jeder die Wahrheit auf seiner
Seite zu haben: beim Fortgang werden beide
zweifelhaft: das Ende soll eben erst die Wahrheit
ausmachen, bestätigen.}
\de{Also darauf hat sich die
Dialektik nicht einzulassen: so wenig wie der
Fechtmeister berücksichtigt, wer bei dem Streit,
der das Duell herbeiführte, eigentlich Recht hat:
treffen und parieren, darauf kommt es an, eben so
in der Dialektik: sie ist eine geistige Fechtkunst;
nur so rein gefaßt, kann sie als eigne Disziplin
aufgestellt werden: denn setzen wir uns zum
Zweck die reine objektive Wahrheit, so kommen
wir auf bloße Logik zurück; setzen wir hingegen
zum Zweck die Durchführung falscher Sätze, so
haben wir bloße Sophistik.}
\de{Und bei beiden würde
vorausgesetzt sein, daß wir schon wüßten, was
objektiv wahr und falsch ist: das ist aber selten
zum voraus gewiß.}
\de{Der wahre Begriff der Dialektik
ist also der aufgestellte: geistige Fechtkunst zum
Rechtbehalten im Disputieren, obwohl der Name
Eristik passender wäre: am richtigsten wohl
Eristische Dialektik: Dialectica eristica.}
\de{Und sie
ist sehr nützlich: man hat sie mit Unrecht in
neuern Zeiten vernachlässigt.}


\de{Da nun in diesem Sinne die Dialektik bloß eine
auf System und Regel zurückgeführte Zusammen-
fassung und Darstellung jener von der Natur ein-
gegebnen Künste sein soll, deren sich die meisten
Menschen bedienen, wenn sie merken, daß im
Streit die Wahrheit nicht auf ihrer Seite liegt, um
dennoch Recht zu behalten; – so würde es auch
dieserhalb sehr zweckwidrig sein, wenn man in
der wissenschaftlichen Dialektik auf die objektive
Wahrheit und deren Zutageförderung Rücksicht
10nehmen wollte, da es in jener ursprünglichen und
natürlichen Dialektik nicht geschieht, sondern
das Ziel bloß das Rechthaben ist.}
\de{Die wissen-
schaftliche Dialektik in unserm Sinne hat dem-
nach zur Hauptaufgabe, jene Kunstgriffe der
Unredlichkeit im Disputieren aufzustellen und zu
analysieren: damit man bei wirklichen Debatten
sie gleich erkenne und vernichte.}
\de{Eben daher muß
sie in ihrer Darstellung eingeständlich bloß das
Rechthaben, nicht die objektive Wahrheit, zum
Endzweck nehmen.}


\de{Mir ist nicht bekannt, daß in diesem Sinne
etwas geleistet wäre, obwohl ich mich weit und
breit umgesehn habe: 9 es ist also ein noch
unbebautes Feld.}
\de{Um zum Zwecke zu kommen,
müßte man aus der Erfahrung schöpfen, beachten,
wie, bei den im Umgange häufig vorkommenden
Debatten, dieser oder jener Kunstgriff von einem
und dem andern Teil angewandt wird, sodann die
unter andern Formen wiederkehrenden Kunst-
griffe auf ihr Allgemeines zurückführen, und so
gewisse allgemeine Stratagemata aufstellen, die
dann sowohl zum eignen Gebrauch, als zum
Vereiteln derselben, wenn der Andre sie braucht,
nützlich wären.}


\de{Folgendes sei als erster Versuch zu betrachten.}



\end{document}
