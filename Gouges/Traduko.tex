% set spell spelllang=eo
\documentclass{book}
\usepackage[utf8x]{inputenc}
\usepackage[english]{babel}
\usepackage{xcolor}
\newcommand{\fr}[1]{{\color{red}#1}\\}
%\newcommand{\fr}[1]{ }
\newcommand{\eo}[1]{#1\\}
%\newcommand{\eo}[1]{ }
\newcommand{\eble}[1]{{\color{blue}#1}}
%\newcommand{\eble}[1]{#1}

\title{Deklaracio pri Virinaj kaj Civitaninaj Rajtoj}
\author{Olympe de Gouges\\Tradukis Ludo}
\date{}

\begin{document}
\maketitle

\indent Al la reĝino

Sinjorino,

\eble{Malmulte} kutimigita al la lingvaĵo ke oni uzas kun regoj, mi ne utiligos
la laŭdado  de la korteganoj por dediĉi al vi tion eksterordinaran
produkto. Mia celo, Sinjorino, estas de sincere paroli al vi; mi ne
atendis, por tiel paroli, la epokon de la libereco: mi min montris kun la
sama energio, en tempo en kio la blindeco de la despotoj punis tia noblan
aŭdacon.

Kiam la tuta imperio kulpigis vin, kaj \eble{responsigis} vin pri siaj
katastrofoj, nur mi, en tempo de tumulto kaj ŝtormo, havis la fortecon de
defendi vin. Mi neniam povis min persvadi ke reĝidino, edukita meze de
\eble{grandanimecoj}, havus ĉiuj la malvirtoj de la malnobleco.

Jes, Sinjorino, kiam mi vidis la glavo direktita al vi, mi ĵetis miajn
observojn inter tiu glavo kaj ties viktimo; sed nun, vidante ke oni
atente observas la amason de ribeluloj subaĉetita, kaj ke ĝi estas
retenita de la timo de la leĝoj, mi diros al vi, Sinjorino, kion mi ne
estus dirinta al vi tiam.

Se la fremdulo \eble{tenas la glavon} en Francujo, vi ne plu estas, laŭ
miaj okuloj, tio erare akuzita reĝino, tio interesa reĝino, sed
malpardonema malamikino de la francoj. Aĥ! Sinjorino, konsideru ke vi
estas patrino kaj edzino; Uzu vian tutan \eble{krediton} por la reveno de
la reĝidoj. Tio \eble{kredito}, tiel saĝe uzita, plifortigas la kronon de
la patro, ĝin konservas por la filo, kaj reamikigas vin kun la amo de la
francoj. Tio digna traktado estas la vera devo de reĝino. La intrigo,
la kabalo, la sangama projektoj hastus vian falon, se oni povus vin
suspekti ke vi kapablas havi tia intencojn.

Pli nobla ofico, Sinjorino, karakterizu vin, kaj altiru vian rigardon.
Nur ŝi, kiun la hazardo nomumis al eminenta rango, devas pezigi la
impeton de la inaj rajtoj, kaj plirapidigi ĝiajn sukcesojn. Se vi estus
malpli instruista, Sinjorino, mi povus timi, ke viaj personaj profitoj
pli gravas por vi ol tiuj de via sekso. Vi ŝatas gloro: konsideru,
Sinjorino, ke la plej grandaj krimoj estas memoritaj kiel la plej grandaj
virtoj; sed, \eble{kio} malsameco \eble{je} famo en la pompoj de la
historio! Unu estas senĉese \eble{ekzempligita}, kaj la alia estas eterne
la abomeno de la homa genro.

Oni neniam konsideros \eble{krima}, ke vi laborus por la restarigo de la
moroj, por doni al via sekso la tutan solidecon ke ĝi indas. Ne estas
\eble{unutaga} laboro, malfeliĉe por la nova reĝimo. Tio revolucio nur
okazos kiam ĉiuj inoj profunde konscios pri ilia plorinda sorto, kaj de
la rajtoj ke ilia perdis en la socio. Subtenu, Sinjorino, tiom belan
kaŭzon; defendu tion malfeliĉan sekson, kaj vi baldaŭ havas por vi, duono
de la reĝlando, kaj almenaŭ triono de la alia.

Jen, sinjorino, jen per kioj prodaĵoj vi devas distingiĝi kaj uzi vian
\eble{krediton}. Kredu min Sinjorino, nia vivo estas ja malgrandvalora,
ĉefe por reĝino, kiam tio vivo ne estas plibeligita de la amo de la
popoloj, kaj de la eternaj ĉarmoj de la bonfarado.

Se veras, ke francoj armas kontraŭ ilia Patrio ilian tutan potencon;
kial? Por frivolaj prerogativoj, por ĥimeroj. Kredu, sinjorino, se mi
juĝdecidas depende de kion mi \eble{sentas}, ke la monarĥia partio sin
mem detruis, ke ĝi forlasos ĉiujn tiranojn, kaj ke ĉiuj korojkuniĝos
ĉirkaŭ la Patrio por defendi ĝin.

Jen sinjorino, jen miaj principoj. Parolante al vi de mia Patrio, mi
perdas el la okulo la celon de tio dediĉo. Tiel ĉio bona ŝtatano
foroferas sia gloro, siaj profitoj, kiam ri nur havas tiujn de sia lando
kiel motivoj.

Mi estas kun la plej profunda respekto, Sinjorino, via tre humila kaj tre
obeema servistino,

\vspace{2em}
\noindent\textbf{La virinaj rajtoj}
\vspace{1em}

Viriĉo, ĉu vi kapablas estis justa? Estas virino kiu faras al vi tion
demandon; vi almenaŭ ne forprenos al ŝi tion rajton. Diru al mi? Kiu
donis al vi la suverena povo opresi mian sekson? Via forteco? Viaj
talentoj? Observu la kreinto en sia saĝeco; Trairu la naturon en ĝian
tutan grandeco, de kio vi ŝajnas voli proksimiĝi, kaj donu al mi, se
vi aŭdacas, ekzemplo de tio tirana povo\footnote{De Parizo ĝis la
Peruo, de la Japano ĝis Romo, la plej stulta besto, laŭ mi, estas la homo}.

Rigardu la bestoj, konsultu la elementoj, studu la vegetaloj, ekrigardu
fine ĉiuj modifoj de la organika materio; kaj konsentas fronte al
evidenteco, kiam mi donas al vi rimedo; serĉu, esploru, kaj distingu, se
vi povas, la seksoj en la administro de la naturo. Ĉie, vi trovos ilin
intermiksitaj, Ĉie ili kunlaboras, kun harmonia aro, por tiu senmorta
mirlaboro.

Nur la homo sin \eble{fuŝvestis principon de tiu escepto}. Stranga,
blinda, dikigita de scienco kaj putrita, en tiu jarcento de klereco kaj
sagaceco, en la plej senlima scivakuo, li volas despote estri sekson kiu
ricevis ĉiujn intelektajn potencojn, li postulas ĝui la Revolucio, kaj
\eble{reklamacii} siaj rajtoj \eble{al} la egalrajteco, por nenion pli
diri.

\vspace{2em}
\noindent\textbf{Deklaracio pri Hominaj kaj Civitaninaj Rajtoj}
\vspace{1em}

Dekretenda de la \eble{Nacia Asembleo} en ĝiaj lastaj seancoj, aŭ en tiuj
de la venonta parlamenta periodo.

\vspace{1em}
\indent\indent\indent\indent Antaŭparolo
\vspace{0.5em}

La patrinoj, la filinoj, la fratinoj, elektitoj de la nacio, petas \eble{esti
konstituitaj} en \eble{Nacia Asembleo}. Konsideranta ke la nekonado, la
forgeso aŭ la indiferenteco pri la virinaj rajtoj estas la solaj kialoj de
la publikaj malfeliĉoj kaj de la koruptado de la registaroj, decidis
eksponi en solena deklaracio, la denaska, necedebla kaj sanktaj rajtoj de
la virino, por ke tiu deklaracio, konstante gravurita en la spirito de ĉiuj
anoj de la sociaro, senĉese rememorigu ilin pri iliaj rajtoj kaj devoj,
por ke la aktoj de la ina povo, kaj tiuj de la iĉa povo, tuttempe
komparebla kun la cela de ĉiu politika establado, estu pli respektitaj, por
ke la reklamoj de la civitaninoj, nun bazita sur simplaj kaj nekontesteblaj
principoj, kontribuu al la konservado de la Konstitucio, de la bonaj moroj,
kaj de la feliĉo de ĉiuj.

Konsekvence, la supera sekso, je la beleco, kiel je la kuraĝo en la
patrinaj suferoj, rekonas kaj deklaras, en la ĉeesto kaj sub la aŭspicioj
de la supera Ŝtato, la ontajn virinajn kaj civitinajn Rajtojn.

\vspace{1em}
\indent\indent\indent\indent Artikolo unua.
\vspace{0.5em}

La virina naskiĝas libera kaj restadas egala al la viriĉo je rajtoj. La
sociaj distingoj povas bazi nur sur la komuna utileco.


\vspace{1em}
\indent\indent\indent\indent II
\vspace{0.5em}

La celo de ĉiu politika asocio estas la konservado de la virinaj kaj
viriĉaj naturaj kaj nepreskripteblaj rajtoj: tiuj rajtoj estas la libereco,
la proprieto, la sekuro, kaj ĉefe la rezisto al la opreso.

\vspace{1em}
\indent\indent\indent\indent III
\vspace{0.5em}

La ĉefa elemento de ĉiu suvereneco estas la nacio, kiu nur estas la kunigo
de la virino kaj de la viriĉo: neniu korpo, neniu individuo povas uzi
aŭtoritato kiu el tiu ne klare emanas.

\vspace{1em}
\indent\indent\indent\indent IV
\vspace{0.5em}

La libereco, kaj la justeco konsistas en redoni kiu apartenas al la aliaj;
jen do la ekzerco de la virinaj naturaj rajtoj nur havas kiel limojn la
eternan tiranion ke la viriĉo kontraŭmetas; tiuj limoj devas esti reformita
de la naturaj kaj raciaj leĝoj.

\vspace{1em}
\indent\indent\indent\indent V
\vspace{0.5em}

La naturaj kaj raciaj leĝoj malpermesas ĉiun agon kiu malutilas al la
socio: ĉiu kiun ne malpermesas tiuj leĝoj, saĝaj kaj diaj, ne povas esti
malebligita, kaj neniu povas esti devigita fari kiu ili ne ordonas.

\vspace{1em}
\indent\indent\indent\indent VI
\vspace{0.5em}

La leĝo devas esti la esprimo de la ĝenerala volo; ĉiuj civitaninoj kaj
civitaniĉoj devas konkuri persone aŭ per iliaj elektitoj, al ties formado;
ĝi devas estis same por ĉiuj: ĉiuj civitaninoj kaj ĉiuj civitaniĉoj, egalaj
laŭ sia okuloj, devas egale akceptita esti al ĉiuj ĉefoficoj, lokoj kaj
publikaj laboroj, kaj sen distingo krom ties virtoj kaj talentoj.

\vspace{1em}
\indent\indent\indent\indent VII
\vspace{0.5em}

Neniu virino estas esceptita; ŝi estas akuzita, arestita kaj tenita en
la okazoj fiksitaj de la leĝo. La virinoj obeas kiel la viriĉoj al tiu
severa leĝo.

\vspace{1em}
\indent\indent\indent\indent VIII
\vspace{0.5em}

La leĝo nur devas fiksi kondamnojn strikte kaj evidente necesajn, kaj
ĉiu povas esti punita nur pro leĝo fondita kaj ekefikigita antaŭ la
delikto, kaj laŭleĝe aplikata al la virinoj.

\vspace{1em}
\indent\indent\indent\indent IX
\vspace{0.5em}

Ĉiu virino estanta kulpigita, ĉiu severeco estas aplikita de la leĝo.

\vspace{1em}
\indent\indent\indent\indent X
\vspace{0.5em}

Neniu devas esti maltrankviligita pro siaj opinioj, eĉ fundamentaj, la virino rajtas iri al la eŝafodo; ŝi devas ankaŭ rajti iri al la tribuno; kondiĉe ke ŝiaj manifestadoj ne malkvietigas la publikan ordon fiksitan de la leĝo.

\vspace{1em}
\indent\indent\indent\indent XI
\vspace{0.5em}

La libera komunikado de la pensoj kaj opinioj estas unu el la plej preciozaj
rajtoj de la virino, ĉar tiu libereco asekuras la legitimecon de la patroj je la
filoj. Ĉiu civitanino do povas libere diri, \textit{mi estas la patrino de
infano, kiu apartenas al vi}, sen ke barbara antaŭjuĝo devigas ŝin, kaŝi la
vero; sed oni devas respondeci pri la misuzo de tiu libereco en la okazoj
fiksitaj de la leĝo.

\vspace{1em}
\indent\indent\indent\indent XII
\vspace{0.5em}

La garantio de la virinaj kaj civitaninaj rajtoj necesas grandegan utilecon; tiu garantio devas ekesti por la profito de ĉiu, kaj ne nur por la privata utileco de tiuj al kiu ĝi estas konfidita.

\vspace{1em}
\indent\indent\indent\indent XIII
\vspace{0.5em}

Por la konservo de la publika ordo, kaj por la administraj elspezoj, la kontribuoj de la virino kaj de la viriĉo egalas; ŝi partoprenas en ĉiuj servutoj, en ĉiuj penigaj taskoj; ŝi do devas ankaŭ partopreni en la disdivido de la lokoj, de la okupoj, de la oficoj, de la rangoj kaj de la industrio.

\vspace{1em}
\indent\indent\indent\indent XIV
\vspace{0.5em}

La civitaninoj kaj civitaniĉoj rajtas mem konstati, aŭ konstati per iliaj mandatitoj, la neceso de la publika kotizo. La civitaninoj povas aliĝi al ĝi nur per la akcepto de egala divido, ne nur de la riĉaĵoj, sed ankoraŭ de la publika administrado, kaj de fiksi la kvoto, la bazo kal la daŭro de la imposto.

\vspace{1em}
\indent\indent\indent\indent XV
\vspace{0.5em}

La amaso de la virinoj, koaliciitaj por la kontribuo al tiu de la viriĉoj, rajtas postuli al ĉiu publika agento, raporti pri sia administro.

\vspace{1em}
\indent\indent\indent\indent XVI
\vspace{0.5em}

Ĉiu socio, en kiu la garantio de la rajtoj ne estas asekurita, nek estas la divido de la povoj fiksita, ne havas Konstitucion; la Konstitucion estas nula, se la plimulto de la homoj kiu partoprenas la nacion ne kunagis en ĝia redakto.

\vspace{1em}
\indent\indent\indent\indent XVII
\vspace{0.5em}

La apartenaĵoj estas al ĉiuj seksoj kunigitaj aŭ dividitaj; ili estas por ĉiu rajto neprofanebla kaj sakrala; neniu povas esti senigita de ili kiel vera hereda propraĵo de la naturo, krom kiam la publika neceseco, laŭleĝe konstatita, evidente postulas ĝin, kaj kondiĉe de justa kaj antaŭa kompenso.

\vspace{1em}
\indent\indent\indent\indent Postparolo
\vspace{0.5em}

Virino, vekiĝu; la alarmo de la racio aŭdiĝas en la tuta
universo; rekonu viajn rajtojn. La potenca influo de la naturo ne
plu estas ĉirkaŭata kun antaŭjuĝoj, fanatikeco, superstiĉoj kaj
mensogoj. La torĉo de la vereco disigis ĉiujn nubojn de la
stulteco kaj de la uzurpado. La sklaviĉo multobligis siajn
fortojn, bezonis uzi viajn por rompi siajn ĉenojn. Liberigita, li
devenis maljusta kontraŭ sia kunulino. Ho virinoj! Virinoj, kiam
vi ĉesos esti blindaj? Kiojn avantaĝojn vi kolektis el la
Revolucio? Malestimo pli markata, malŝato pli signalata. En la
jarcentoj de koruptado, vi nur regis la malfortecon de la
viriĉoj. Via influo estas detruita; kio do restas al vi? La
certeco de la maljustecoj de la viriĉoj. La reklamo de via
heredaĵo, fondita sur la saĝaj dekretoj de la naturo; kion do vi
timis por tiom bela iniciato? La bona vorto de la leĝfaranto de
la Kanaa geedziĝo? Ĉu vi timas ke niaj francaj leĝfarantoj,
korektistoj de tiu moralo, longtempe fiksita al la branĉoj de la
politiko, sed kiu ne plu estas sezona, ripetas al vi: virinoj,
kiu estas komuna inter vi kaj ni? Ĉiu, vi devus respondi. Se ili
obstine daŭrigis, en ilia malforteco, konfliktigi tiun
malkoheraĵon kun iliaj principoj; kuraĝe starigu la fortecon de
la racio al la vanaj pretendoj de supereco; kuniĝu sub la flago
de la filozofio; disetendu la tutan energion de via karaktero,
kaj vi baldaŭ vidos tiujn malhumilulojn, niaj sklavaj adorantoj
rampantaj antaŭ viaj piedoj, sed fieraj kundividi kun vi la
trezorojn de la Supera-Estaĵo. Kiu ajn la barojn ke oni starigas
kontraŭ vi, estas en via povo transiri ilin; sufiĉas ke vi volu
tion. Nun ni parolu pri la terura bildo de kion vi estis en la
socio; kaj ĉar temas nuntempe pri nacia edukado, ni vidu ĉu niaj
saĝoj leĝfarantoj sane pensis pri la edukado de la virinoj.

La virinoj faris pli da malbonon ol bonon. Trudo kaj kaŝemo estis
ilia atribuajo. Kion la forteco forrabis de ili, la ruzo redonis
al ili; ili uzis ĉiujn rimedojn de sia ĉarmo kaj la plej
senriproĉuloj ne malcedis al ili. La veneno, la fero, ĉiu estis
submetita al ili; ili estris krimon kiel virton. La franca
registaro ĉefe, dependis dum jarcentoj de la nokta administrado
de la virinoj. La kabineto havis neniujn sekretojn por ilia
nediskreteco; ambasadoreco, estreco, ministreco, prezidanteco,
papeco\footnote{s-riĉo de Bernis, maniere de s-rino de
Pompadour}, kardinaleco; mallonge, ĉiu kiu karakterizigas la
stultecon de la viriĉoj, profana kaj sakra, ĉiu estis submetita
al la avareco kaj ambicio de tiu sekso, tiam malestiminda kaj
respektita, kaj ekde la Revolucio, respektinda kaj malestimata.

En tia antitezo, kiom da komentariojn mi havas ofereblaj! Mi nur
havas momenton por fari ilin, sed tiu momento fiksos la atenton
de la plej malfrua posteularo. Dum la malnova reĝimo, ĉiu estis
perversa, ĉiu estis kulpa; sed ĉu oni ne povus ekvidi plibonigon
de la afero en la eĉ substanco de la malvirtoj? Virino nur
bezonis esti bela aŭ aminda; kiam ŝi havis tiujn du avantaĝojn,
ŝi vidis cent fortunojn antaŭ siaj piedoj. Se ŝi ne profitis
ilin, ŝi havis strangan karakteron, aŭ nekutiman filozofion, kiu
kondukas ŝin al malestimo de la riĉaĵoj; tiam, ŝi plu estis
konsiderata nur kiel nekonformisto; la plej maldeca sin
respektigis kun oro; la virintrafiko estis ia industrio ricevita
en la unua klaso, kiu nuntempe ne plu havos krediton. Se ĝi plu
havus krediton, la Revolucio estus perdita, kaj kun novaj
rilatoj, ni daŭre estus koruptata; tamen ĉu la racio povas kaŝi
al si ke ĉiu alia vojo al la fortuno estas fermata al la virino
kiun la viriĉo aĉetas, kiel la sklavo sur la afrikaj marbordoj?
La malsameco estas granda; oni tion scias. La sklavo estras la
mastron; sed se la mastro donas al ŝi la liberecon sen
rekompenco, kaj je aĝo en kio la sklavo malgajnis ĉiujn siajn
ĉarmojn, kio devenas tiun malfortunulino? La ludilo de la
malestimo; eĉ la pordoj de la bonfaremo fermiĝas antaŭ ŝin; ŝi
estas malriĉa kaj maljuna oni diras; kial ŝi ne sukcesis riĉiĝi?
Aliaj ekzemploj eĉ pli kompatvekaj sin prezentas al la racio.
Juna malsperta, delogita de viriĉo ke ŝi amas, forlasos siajn
gepatroj por sekvi lin; la sendankulo lasos ŝin post kelkaj
jaroj, kaj ju pli ŝi estos aĝita kun li, des pli lia malstabileco
estos malhoma; se ŝi havas infanojn, li same forlasos ŝin. Se li
estas riĉa, li pensos esti maldevigita disdividi sian fortunon
kun siaj noblaj viktimoj. Se ia devontigo ligas lin al liaj
devoj, li malobeos ties potenco, esperante ĉiun de la leĝoj.
Se li estas geedziĝita, ĉiu alia devontigo malgajnas siajn
rajtojn. Kiu leĝo do restas farendo por eltiri la malvirton eĉ el
la radiko? Tiu de la disdivido de la fortunoj inter la viriĉoj
kaj la virinoj, kaj de la publika administrado. Oni facile imagas
ke ŝi kiu naskiĝis en riĉa familio multe gajnas per la egaleco de
la disdivido. Sed ŝi kiu naskiĝis en malriĉa familio, kun indo
kaj virtoj, kiu estas ŝia asignaĵo? Malriĉeco kaj malgloro. Se ŝi
ne precize spertas muzikon aŭ pentrarton, ŝi povas esti
akceptata en neniu publika ofico, eĉ se ŝi tion tute kapablas. Mi
nur volas doni bildon de la afero, mi pli esploris ĝin en la nova
eldono de ĉiuj miaj politikaj verkoj kiujn mi intencas doni al la
publiko post kelkaj tagoj, kun miaj notoj.

Mi rekomencas mian tekston pri la moroj. La geedziĝo estas la
tombo de la fido kaj de la amo. La edziniĝita virino povas
senpune doni bastardojn al ŝia edziĉo kaj la fortuno kiu ne
apartenas al ili. Tiu kiu ne estas edziniĝita nur havas
malgrandan rajton: la leĝoj malnovaj kaj malhomaj rifuzis al ŝi
tiun rajton koncerne la nomo kaj la havaĵoj de ilia patro por
ŝiaj infanoj, kaj oni ne faris novajn leĝojn koncerne ĉi tio. Se
provi doni al mia sekso honora kaj justa firmeco estas
konsiderata en tiu momento kiel paradokso miaflanke, kaj kiel
provi la nefareblon, mi lasas al la venontaj viriĉoj la gloro de
trakti ĉi tiun temon; sed, ĝis tiam, oni povas prepari ĝin per
nacia edukado, per la restarigo de la moroj kaj per la edzecaj
akordoj.

\vspace{2em}
\noindent\textbf{Formo de la socia kontrakto de la viriĉo kaj de
la virino}
\vspace{1em}

Ni N kaj N, el nia propra volo, kuniĝas por la fino de nia vivo,
kaj por la daŭro de niaj reciprokaj inklinoj sub la sekvaj
kondiĉoj: Ni konsentas kaj volas komunumigi niajn fortunojn,
gardantaj tamen la rajto disigi ilin favore al niaj infanoj, kaj
de tiuj kiujn ni povus havi el speciala inklino, reciproke
agnoskanta ke niaj havaĵoj rekte apartenas al niaj infanoj, de
kiu ajn lito ili venas, kaj ke ĉiuj sendistinge rajtas uzi la
nomon de la patriĉoj kaj patrinoj kiuj rekonis ilin, kaj ni nin
trudas konsenti pri la leĝo kiu punas la malagnosko de sia propra
sango. Ni ankaŭ nin devigas, okaze de disigo, fari la disdividon
de nia fortuno, kaj elpreni la parton de niaj infanoj priskribata
de la leĝo; kaj, okaze de perfekta kuniĝo, tiu kiu mortas
forlasos duonon de siaj propraĵoj favore al siaj infanoj; kaj se
unu mortus sen infanoj, la postvivanto laŭleĝe heredus, krom se
la mortanto donis duonon de la komunajn riĉajojn favore al kiu li
juĝis justa.

Tio estas proksimume la formulon de la edzeca akto kies plenumon
mi proponas. Leganta tiun strangan skribon, mi antaŭvidas
kontraŭstari min la tartufojn, la prudoj, la klerikaro kaj ĉiun
la infernan sekvon. Sed kiom ĝi oferos al la saĝoj da moralajn
rimedojn por atingi la perfekigtebleco de feliĉa estraro! Mi
donos malmultvorte ties fizikan pruvon. La riĉa Epikurano sen
infanoj, trovas tre agrabla iri ĉe lia malriĉa najbajro pliigi
lian familion. Kiam estos leĝo kiu permesas al la edzino de la
malriĉulo igi la riĉulon adopti ŝiajn infanojn, la ligoj de la
socioj estos pli premataj, kaj la moroj pli puraj. Tiu leĝo eble
konservos la bonon de la komunumo, kaj retenos la malordon kiu
kondukas tiom da viktimoj en la hospicoj de la malgloro, de la
fieco kaj de la kadukiĝo de la homaj principoj, kie, jam de
longtempe, ĝemas la naturo. La kontraŭuloj de la sana filozofio
do ĉesu protesti kontraŭ la praaj moroj, aŭ ili perdiĝu en la
fonto de iliaj citaĵoj\footnote{Abraham havis infanojn tre
laŭleĝaj de Agar, servistino de lia edzino}

Mi ankoraŭ volas leĝon kiu favorus la vidvinoj kaj la fraŭlinoj
trompitaj de la malveraj promesoj de viriĉo al kiu ili estus
korinkliniĝitaj; mi volas, mi diras, ke tiu leĝo igu malfirmuliĉo
plenumi liajn devigojn, aŭ pagi kompenson proporcia al lia
fortuno. Mi ankaŭ volas ke tiu leĝo estu severa kontraŭ la
virinoj, almenaŭ tiuj kiuj aŭdacus sin helpi per leĝo kiun ili
mem malobeis per ilia malbonkonduto, se tio estas pruvita. Mi
volas, samtempe, kiel mi prezentis en \textit{Le Bonheur primitif
de l'Homme}\footnote{La praa feliĉo de la homo (noto de la
tradukisto)} en 1788, ke la publikulinoj estu instalataj en
specifaj kvartaloj. Ne la publikulinoj plej kontribuas al la
diboĉigo de la moroj, sed la sociulinoj. Restariganta la lastajn,
oni ŝanĝas la unuajn. Tiu ĉeno de frata kuniĝo unue oferos la
malordon, sed poste, ĝi fine produkis perfektan aron.

Mi oferas nevinceblan manieron por altigi la animon de la
virinoj; estas kunigi ilin al ĉiuj oficoj de la viriĉo. Se la
viriĉo obstine trovas tion manieron nepraktika, li disdividu lian
fortunon kun la virino, ne laŭ liaj kapricoj sed per la saĝeco de
la leĝoj. La antaŭjuĝo falas, la moroj purigiĝas, kaj la naturo
reprenas ĉiujn siajn rajtojn. Aldonu la geedzecon de la
pastriĉoj; la Reĝiĉo, firmigata sur lia trono, kaj la franca
registaro ne plu povus perei.

Vere necesis ke mi diru kelkajn vortojn pri la ĝenoj kiujn
kaŭzas, laŭdire la dekreto favora al la koloraj homoj, en niaj
insuloj. Tie la naturo horore tremas, tie la racio kaj la homeco
ne jam tuŝis la malmolajn animojn; tie ĉefe la divido kaj la
malakordo agitas iliajn loĝantojn.

\end{document}





